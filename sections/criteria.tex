\section{Benchmark metrics}
The quality of the estimated poses and thus of the applied algorithm is assessed by metrics that define how to measure the deviation of the prediction from ground truth. The common approach is to first extract the features from an estimated pose, which are then compared with the features calculated from ground truth. The measured values can be used to assess how precise the pose estimate is. For this analysis, the previously mentioned datasets from \autoref{sec:benchmarks} are used as reference. Some metrics have their own thresholds that define whether an estimate is correct or deviates too much from reality and is therefore false.

\subsection{Percentage of Correct Parts PCP}
\label{sec:pcp}
This criterion encompasses a comparison of the recognized and recognizable body parts. The definition of a correctly recognized limb includes both the distances $l_{1}, l_{2}$ of its endpoints from those contained in the dataset and its total length $L$. \autoref{fig:pcp} shows an example with the left (green) line indicating information from the dataset that is compared to the estimated right line. Another factor $p$ multiplied by $L$ defines the threshold value to which $l_{1}$ and $l_{2}$ are compared. If $l_{1}$ or $l_{2}$ exceed the threshold, the body part is not detected correctly, resulting in a lower $PCP$ score.
The smaller p, the stricter the evaluation and thus higher the accuracy.\cite{pcp}

\begin{figure}[h]
\centering
\includegraphics[]{"criteria/pcp.png"}
\caption{Abstract visualisation of the values required for the PCP calculation. A solid line between two endpoints represents a bone in the ground truth of a dataset in green and the detected bone in black. The dashed lines show the distances $\protect l_{1}, l_{2}$, which are evaluated with the true length $\protect L$ of the bone.}
\label{fig:pcp}
\end{figure}

\subsection{Percentage of Detected Joints PDJ}
\label{sec:pdj}
The metric \emph{PCJ} addressed in this section is similar to the already established \emph{PCP} methode. While \emph{PCP} depends on individual limb lengths, \emph{PCJ} uses the torso length $T$ as a global reference, so that a body part is correctly detected if both $l_{1}$ and $l_{2}$ do not surpass the threshold given by $T$ and a factor $p$. \cite{pdj}

\subsection{Percentage of Correct Key-points PCK}
For this particular criteria, the maximum bounding box length $B$ must be calculated from the existing dataset information. The \emph{PCK} score is calculated analogously to the PCP and PDJ metrics, where the threshold is defined by $B \cdot p$. \cite{pck}

\subsection{Head-normalized Probability of Correct Key-points PCKh}
In \emph{PCKh}, the following change to the standard \emph{PCK} was proposed by setting the threshold to $H \cdot 0.5$, which corresponds to half of the head frame. Since the maximum length of the bounding box depends on the position of the individual body parts, the head box is chosen so that the threshold is not influenced by the subjects' articulation. \cite{mpii}


%\subsection{Area Under the Curve AUC}
%The measurement of the \emph{PCK} under variation of the threshold value results in a %curve. This analysis provides information on how the model is able to distinguish the %individual joints of the body. A large curve area defines a qualitative model. 
%\cite{auc}

\subsection{Object Key-point Similarity OKS}
OKS measures the overlap between the extracted keypoints of the pose estimate and the grount truth and thus whether the predicted keypoints are close to reality.
\autoref{eq:OKS} illustrates the simplified OKS score\footnote{\url{https://cocodataset.org/\#keypoints-eval}}, which is a sum of $ n \in \mathbb{N}$ detected and ground truth key-points. The parameters $d \in \mathbb{R}$ are the Euclidean distance between the corresponding ground truth value and the detected key-point, while $s \in \mathbb{R}$ denotes the object segment area and $k \in \mathbb{R}$ is a constant describing a falloff. 

\begin{equation}
    \label{eq:OKS}
    OKS = \sum_{i=1}^{n} e^{- \frac{d_{i}^{2}}{2 \times s^{2} \times k_{i}^{2}}}
\end{equation}
Optimal predictions have a high OKS value, while low values indicate poor predictions. 

\subsection{Mean Per Joint Position Error (MPJPE)}
\label{criteria:mpjpe}
\autoref{eq:MPJPE} shows the formula for one of the most import criteria in terms of evaluating results from 3d human pose estimation the Mean Per Joint Position Error or short MPJPE \cite{Zheng2020}. $N$ denotes the number of estimated joint positions $J_{i}^*$ and ground truth joint positions $J_{i}$. 

\begin{equation}
\label{eq:MPJPE}
MPJPE = \frac{1}{N} \sum_{i=1}^{N} \lVert J_{i} - J_{i}^*\rVert_{2}
\end{equation}

There are versions of this criteria too, such as PMPJPE, where the estimated pose gets rigidly aligned to the ground truth first. NMPJPE stands for Normalized MPJPE, what means that the estimation gets normalized to scale of the true joint positions. 