\section{conclusion}
As research in the topic of human pose estimation is increasing, it becomes more important to have an overview about methods used in this field. Therefore, this paper presented selected works of the last years. First commonly used datasets were described, followed by metrics for benchmarking to be able to compare different methods against each other. After that, as a representative of models in (3d) human pose estimation, SMPL was introduced in \autoref{sec:SMPL}. 
\\2d pose estimation was next. Since neural networks are currently the most powerful tools in this area, challenges when designing those have to be discussed first. The top-down approach, which divides input images such that only a single person appears on one part, was now further investigated. While heat-map-based approaches are the most precise at the moment, regression-based methods with convolutional neural networks are more efficient in scenarios where real-time performance is important, because they are not as computationally expensive compared to the many steps a heat-map method needs. Also, training is easier with regression-based methods, as it can be done in an end-to-end fashion. Unlike top-down methods, bottom-up approaches try to estimate joint positions from an input image first, and then merge them to individual humans. Since the pose is not calculated for each human individual, these methods are faster than their top-down complement for the most part when multiple humans can be found in an image, because the performance of top-down approaches drops with the number of human poses to be estimated increasing. However, if not many persons are to be found (and they dont overlap each other), top-down methods yield higher performance. Future research needs to be focussed to improve computational speed even further to make real-time applications possible, and decrease failure rates in scenarios with many humans or occlusion.