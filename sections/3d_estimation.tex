\section{3d pose estimation}
From a historical perspective, a 3d motion capture algorithm consists of 4 sequential processes: initialisation, tracking, pose estimation and recognition. Initialization involves both camera and model initialization, i.e. setting the camera calibration and finding a model that represents the subject and assigning its initial pose manually or automatically. Model-based approaches can be viewed iteratively, with each frame of the data source representing an iteration in which the initial pose is refined. Tracking is concerned with the relationship between the parts of the subject's body. This leads to segmentation of the subject from the background, representation changes, and establishing tracking in further images. The next phase, which is mainly covered in this section, is the estimation of the pose. A distinction is made between model-based and non-model-based methods, with the former requiring  \emph{a priori} a model. In that approaches, especially human pose estimation, a human model is used to benefit from its encoded information. This model can either be used indirectly, considering e.g. only general aspects such as size and structure, or it is a direct used model. Directly used models are both more detailed and offer broader benefits in regards to occlusion handling and embedded kinematic constraints. In an application, the observed object is approximated by the model, which is continuously refined with further images.\cite{summary80s}

\subsection{SMPL}

