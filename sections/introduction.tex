\section{Introduction}
In computer vision, image data acquisition is divided into the following categories. \emph{passive} or \emph{active} sensing. Passive sensing uses visible light or other electromagnetic wavelengths to acquire position data, while active sensing requires special devices attached to the person's body to obtain parametric data. 
\\\\
As computers get more powerful and more involved with humans, it is important to be able to estimate human positioning accurately. Autonomous cars need to make decisions considering the near future, therefore they need to be able to predict human motion, which is tightly coupled with their pose. The same applies to robotic systems working around or even with humans. Since human posing is an indicator for their emotions aswell, social or collaborative robots could use this information in the future. Human poses can also be used to control systems or play games without the need for peripheral devices. For video surveillance, human poses could be used in the future to help identify individuals, as poses of humans vary from person to person. Even medical information can be acquired from a humans pose and could be used in therapy or for correct posture when sitting or exercising sports. These examples show that human pose estimation is an important research area today and will become even more so in the future. \\
However, there are challenges to overcome in order to deliver accurate estimations. Firstly, there are error sources in pictures or videos taken such as different lightning conditions, backgrounds or clothing. Multiple or incomplete humans could also hamper the algorithm. Since most data is acquired using a 2d camera, 3d pose estimation needs to add an extra dimension of depth, which could lead to ambiguities. As neural networks, which are used a lot for pose estimation today, get bigger and used more frequently, enough and sufficient training and testing data needs to be available in the future. These datasets also need to be representative for humans in their normal environment. Annotation, especially for 3d data, can be a very resource-intensive and time-consuming task too.
\\
This paper aims to summarize metrics methods for human pose estimation in two or three dimensions and show off their strengths and weaknesses. Finally, the current state-of-the-art can be evaluated and an outlook into the possible future can be provided.