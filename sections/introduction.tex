\section{Introduction}
Human body estimation is a broad and increasingly relevant area of research that allows human movements to be detected or even predicted, and also their meshes to be created based on visual input. Due to the widespread distribution and availability of relatively inexpensive but powerful video recording hardware and computers, the researched technique of pose estimation can be used by many people in various fields of application. One such application may be in surveillance areas where installed cameras record information to facilitate later investigations into possible crime, e.g. in the security software of petrol stations or car parks. Or in control-based domains such as  virtual and augmented reality, where immersion is achieved by capturing human movements and executing them through the avatar. This can be applied not only to today's video games, but also to online business meetings, sprints or educational courses that can take place in a virtual environment. Moreover, the approaches are existential for the futuristic meta-project in which social life is also shifting into a virtual world and new forms of communication are being introduced. An further interesting aspect is also the use in autonomous maschines. Here autonomous cars need to make decisions considering the near future, therefore they need to be able to predict human motion, which is tightly coupled with their pose. The same applies to robotic systems working around or even with humans. Since human posing is an indicator for their emotions aswell, social or collaborative robots could use this information in the future. Even medical information can be obtained from a person's posture and used in diagnostics and therapy or for the correct posture when sitting or doing sports. These examples show that human pose estimation is an important research area today and will become even more so in the future. However, there are challenges to overcome in order to deliver accurate estimations. Firstly, a vulnerable source of error is associated with the acquisition of the image data, which is divided into the following categories. \emph{passive} or \emph{active} sensing. 
\\
Passive sensing uses visible light or other electromagnetic wavelengths to acquire position data, while active sensing requires special devices attached to the person's body to obtain parametric 3d data. In the initial phase of active sensing, the object whose movement is to be detected must be wired, which makes its movement cumbersome \cite{mocap}. Newer systems offer more flexibility, but are prone to heading errors that accumulate over the recording period and lead to inaccuracies \cite{vip}.
With passive sensing, errors can occur due to different lighting conditions, backgrounds or clothing. 
\\
Multiple or incomplete humans could also hamper the algorithm. Furthermore, 3d pose estimation needs to add an extra dimension of depth, which could lead to ambiguities. As neural networks, which are used a lot for pose estimation today, get bigger and used more frequently, enough and sufficient training and testing data needs to be available in the future. These datasets also need to be representative for humans in their normal environment. Annotation, especially for 3d data, can be a very resource-intensive and time-consuming task too.
\\
This paper aims to summarize metrics methods for human pose estimation in two or three dimensions and show off their strengths and weaknesses. Finally, the current state-of-the-art can be evaluated and an outlook into the possible future e.g. is provided.