\section{2D pose estimation}
The foremost step in Human Pose Estimation (HPE) is to model the human body entity representing its kinematic structure and shape information. In recent studies, benefiting from the intuitive nature of graph representations, the human body structure has been characterized by anatomical joints and their positions. This approach is known as the kinematic model and is widely employed to describe human poses. In 2D Human Pose Estimation, the goal is to localize the 2D position or spatial location of human body keypoints (kinematic joints) from the input data \cite{rafi_efficient_2016}.

Recently, deep neural networks have achieved a significant breakthrough in 2D HPE and are mainly employed to extract robust features for keypoint recognition and localization directly from the input data (images and videos). Since the quality of feature representation is intimately tied to network architecture, the subject of network design was chosen to be thoroughly investigated. 

\subsection{HPE Network Architecture Design Challenges}
Like many tasks in computer vision (e.g., object detection, image segmentation, etc.), the primary technological challenges in developing 2D HPE algorithms are the accuracy and efficiency of the proposed method. High precision pose estimation facilitates accurate human body information for subsequent tasks, including lifting from 2D to 3D HPE models. 

In addition to the accuracy of an HPE algorithm, its efficiency (inference speed) is also desired for real-time applications. However, there is mostly a trade-off between accuracy and efficiency concerning algorithm development since the high accuracy methods tend to be more complex and demand powerful resources for computation. Consequently, lightweight models with comparable precision are more desirable for mobile or wearable devices. 

Another crucial challenge in 2D HPE is estimating poses for multi-person scenarios. 2D single-person HPE localizes joint coordinates when only one human instance is in the input image, but for images containing multiple persons, the HPE model needs to recognize each human in the input image before estimating its keypoints positions. 

This process can be done by a two-stage process called the top-down approach that first detects individual persons, i.e., the input image is divided into sub-images (patches) enclosing just one person \cite{sun_deep_2019}. Then the deep learning keypoint detection algorithm can be applied to each patch. 

On the other hand, there is an alternative group of approaches known as bottom-up, in which human joints are detected directly from the image without an explicit object detection stage \cite{cheng_higherhrnet_2020}. The following sections will review current frameworks utilizing top-down and bottom-up approaches to their network architectures.

\subsection{Top-Down Approaches}

The top-down approaches employ human body detectors \cite{ren_faster_2016, micilotta_real-time_2006} to obtain a set of bounding boxes (each corresponding to one person) from the input images and then perform pose estimation within each bounding box. The existing literature in 2D HPE based on top-down approaches can also be divided into fine-grained sub-categories discussed in this section: regression-based and heat map-based methods. 

\subsubsection{Regression-based methods}

There are many works based on the regression technique
\cite{carreira_human_2016, fan_combining_2015, fieraru_learning_2018, li_heterogeneous_2014, qiu_peeking_2020, sun_compositional_2017, sun_integral_2018, toshev_deeppose_2014, wang_graph-pcnn_2020,z} that directly regress human body joint coordinates via an end-to-end trained network that maps the input image to the pre-defined graphical structures (kinematic keypoints). Accordingly, the regression hypothesis is defined as follows:

Given an image I, the goal of pose estimation is to predict a possibly empty set of human instances, $\{Pi\}_{i=1}^N$ where \textbf{\textit{N}} is the number of persons in the image. For each person, we need to predict its bounding box information, $\{Bi\}$, as well as its keypoint coordinates, $Ki=\{(x_j, y_j)\}_{j=1}^J$, where \textbf{\textit{J}} is the number of pre-defined joints in each dataset \cite{li_pose_2021}.

DeepPose \cite{toshev_deeppose_2014} is a regressor that explicitly predicts human joint locations through fully connected layers of a deep neural network. AlexNet \cite{krizhevsky_imagenet_2012} as the backbone of DeepPose's cascaded architecture for feature extraction. Due to the remarkable performance of DeepPose in efficient keypoint detection by using convolutional neural networks (CNNs), human pose estimation techniques have gradually migrated from the conventional graphical models to deep learning approaches. 

Carreira et al. \cite{carreira_human_2016} proposed a model based on GoogLeNet [220-2] that adopts an iterative feedback mechanism called Iterative Error Feedback (IEF) to enhance the efficiency of hierarchical feature extractors such as Convolutional Networks (ConvNets) in early layers. This self-correcting model progressively adjusts an initial estimate of joint coordinates by feeding back predictions error in the IEF process instead of directly predicting the coordinates in a feed-forward network.

Transformer models \cite{vaswani_attention_2017} based on the self-attention mechanism have significantly advanced the field of representation learning and promoted visual understanding tasks such as object detection frameworks that are free of region proposals, anchors, and post-processing (non-maximum suppression) procedures. 

In a recent study \cite{li_pose_2021}, authors took full advantage of the tokenized representation in transformers with self-attention layers. They propose a regression-based human pose estimation method, called pose recognition Transformer (PRTR), using two types of cascade transformers based on an end-to-end object detection Transformer (DETR) \cite{carion_end--end_2020}. In this proposed model, the second transformer is a regressor that predicts joints' coordinates within each image patch detected from the first transformer and performs multi-person pose estimation. PRTR reveals competitive results for pose recognition compared with other existing regression-based methods on the challenging COCO dataset.


\subsubsection{Heatmap-based Methods}

In the field of human pose estimation, heatmap-based approaches seek to train a deep neural network to approximate the locations of body parts and joints based on heatmap representations \cite{chen_articulated_2014, newell_stacked_2016, wei_convolutional_2016}. Accordingly, during training, the pose estimation model generates J heatmaps $\{Hj\}_{j=1}^J$, as a 2-dimensional Gaussian distribution centered at the ground-truth joint location, where J is the total number of joints. The pixel value $Hj \{(x_i,y_i)\}$ in each joint heatmap encodes the probability that the $j^{th}$ joint lies in the location $(x,y)$  \cite{tompson_efficient_2015, tompson_joint_2014}. 

The probability values provide richer supervision that facilitates the training of convolutional networks and significantly improves the performance of heatmap-based over regression-based approaches \cite{artacho_unipose_2020, bulat_human_2016,gkioxari_chained_2016, lifshitz_human_2016,newell_stacked_2016,tompson_efficient_2015}. Therefore, heatmaps preserve the spatial location information of joints and reduce the number of false-positive cases, which results in an increasing attraction to develop deep learning pipelines for HPE based on heatmap representations \cite{carreira_human_2016,luo_lstm_2018,toshev_deeppose_2014,wei_convolutional_2016}.

Ramakrishna et. al \cite{ramakrishna_pose_2014} presented a sequential prediction framework based on convolutional networks named Convolutional Pose Machines (CPM) that learns rich implicit spatial structures by utilizing the inference machine framework to incrementally refine estimates of the human body part (e.g., head, leg, etc.) locations in multiple stages. In this method, a predictor is trained to predict the confidence of the body part locations in each stage. In the first stage, the predictors estimate the confidence of each part location from heatmaps computed on the input image patch. Since a heatmap produced only from image features is noisy and has multiple modes, in the subsequent stage, the network iteratively refines the estimated confidences using contextual information from outputs of the previous stage. 

In a further expanding of Pose Machine architecture, \cite{wei_convolutional_2016} employed sequential convolutions to implicitly model long-range spatial relationships between different parts. It is worth mentioning that multiple stages increase the network depth and make it hard to train because of the vanishing gradient problem — as the gradient is back-propagated to the layers of earlier stages, repeated multiplication makes the gradient infinitely small. Thus, as the network goes deeper, its performance gets saturated or degrades rapidly. 

Before ResNet \cite{he_deep_2016}, there were several methods to deal with the vanishing gradient issue; for instance, the authors of this work \cite{wei_convolutional_2016} developed an auxiliary loss in a middle layer as extra supervision. But this can not tackle the problem entirely since each stage will fail to extract robust semantic features from the input data and is prone to overfitting.

The core idea of ResNet is introducing "shortcut connections" that skip (jump over) one or more layers. These skip connections effectively simplify the network and reduce the impact of vanishing gradients as it allows the back-propagation of training errors at deeper levels (addressing the issue of iterative architectures with multiple stages). Residual networks dramatically improved the 2D HPE process, and numerous models
\cite{cai_learning_2020, chen_cascaded_2018, chu_multi-context_2017, ke_multi-scale_2018, liu_cascaded_2018, newell_stacked_2016, su_multi-person_2019, sun_deep_2019, xiao_simple_2018-1, yang_learning_2017} have been developed due to their advantage. An encoder-decoder network called "stacked hourglass" (SHG) based on the sequential stages of pooling and upsampling layers was presented by Newell et al. \cite{newell_stacked_2016} to capture spatial correlations between the human body keypoints by combining low and high-resolution feature representations. 

Several more complicated variants have been introduced following the initial success of SHG architecture; specifically, Chu et al. \cite{chu_multi-context_2017} developed novel Hourglass Residual Units (HRUs) extended by a side branch of filters with larger receptive fields that learn features across various scales. Yang et al. \cite{yang_learning_2017} replaced the residual units in SHG with multi-branch Pyramid Residual Modules (PRMs) to enhance deep convolutional neural networks by constructing scale invariance features. 

Cai et al. \cite{cai_learning_2020} recently proposed the Residual Steps Network (RSN), which efficiently maintains rich low-level spatial information by aggregating intra-level features with the same spatial size to localize the keypoints precisely using delicate local representations. In addition, they devised a novel attention mechanism, Pose Refine Machine (PRM), that refines the keypoint locations by finding a trade-off between the contribution of local and global representations in the resultant feature. Their approach accomplished state-of-the-art results on COCO and MPII benchmarks and won 1st place in the COCO Keypoint Challenge 2019.


\subsection{Top-Down Approaches Summary}

The primary components of top-down frameworks in human pose estimation consist of an object detector and a pose estimator to predict the joint positions of the human body. The object detector determines human proposal detection performance and influences pose estimation. On the other hand, the pose estimator component is the framework's heart that directly affects the pose estimation accuracy. The main reason for adopting heat map-based and regression-based methods for the pose estimator component is the speed-accuracy trade-off. 

In regression-based human pose estimation, the problem is formulated as a regression one in which features extracted from convolution layers of a CNN are regressed to directly predict joint coordinates of the body parts. The regression-based networks can be trainable end-to-end and are efficient in real-time applications since they have fewer intermediate non-differentiable steps \cite{liu_cascaded_2018, eichner_human_2012, felzenszwalb_pictorial_2005, girdhar_detect-and-track_2018}; however, they typically perform less accurately than heat map-based approaches. Because when the body parts are not completely visible, they fail to accurately estimate the locations of human body joints. To address this shortcoming, probabilistic heatmaps are employed to learn a complex mapping from occluded part appearances to joint coordinates instead of directly regressing them. 

Human pose estimation based on heatmaps comprises pre-processing and post-processing procedures to encode keypoints' ground truth (GT) into heatmaps and decode heatmaps to predict joint locations. Consequently, they can be optimized easier and in comparison to regression-based methods, have a more robust generalization that delivers substantial performance, primarily suitable to be adopted when accuracy is the priority \cite{pfister_flowing_2015}. 

However, heatmap-based frameworks have various heuristic network designs that are mostly not end-to-end learnable and suffer from several shortcomings: Firstly, to generate the 2D heatmaps, computationally expensive upsampling operations (e.g., deconvolution layers in \cite{yixing_gao_user_2015}) are required. Also, an extra post-processing step for reducing projection errors from heatmap to ground truth coordinates is unavoidable in further keypoint estimation refinements. This makes them sub-optimal, i.e., the speed efficiency (performance) of heatmap-based methods usually declines drastically as the input image resolution decreases. 

\subsection{Bottom-Up Framework}
 
After analyzing the top-down frameworks in the network architecture design context for human pose estimation, it is noteworthy to consider the bottom-up approaches. These approaches (e.g., [15, 32, 56, 75, 76, 90, 109, 169, 174, 191, 226, all 2]) aim to perform two main tasks: firstly, human keypoint detection through extracting local features from the input image and proposing a set of human body joint candidates, and then grouping those candidates to build pose representations for each individual person. The main distinguishing characteristic of the bottom-up approach from the former approach is that the framework design does not rely on a human detection component to predict human bounding boxes separately. As a result, the computational overhead decreases substantially by directly estimating human poses from extracted features; however, the major challenge is an identification mechanism for estimated keypoints to associate different subsets of candidates to each individual human body. 